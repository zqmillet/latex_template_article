\documentclass[punct = banjiao]{ctexart}
% 中文字体采用 ctex 默认配置,调整英文字体配置
\setromanfont{Times New Roman}
\setsansfont{Arial}

% 数学配置
\usepackage{mathrsfs}
\usepackage{amssymb}
\usepackage{amsmath}
\usepackage{bm}
\usepackage{mathtools}

% 设置段间距
\setlength{\parskip}{0.5\baselineskip}

% 设置标题格式
\usepackage{titlesec}
\titleformat*{\section}{\bfseries\fontsize{14pt}{14pt}\selectfont}
\titlespacing*{\section}{0pt}{10pt}{6pt}
\titleformat*{\subsection}{\bfseries\fontsize{12pt}{12pt}\selectfont}
\titlespacing*{\subsection}{0pt}{10pt}{6pt}
\titleformat*{\subsubsection}{\bfseries\fontsize{12pt}{12pt}\selectfont}
\titlespacing*{\subsubsection}{0pt}{10pt}{6pt}
\setcounter{secnumdepth}{4}
\setcounter{tocdepth}{1}
  
% 配置页面布局
\usepackage{geometry}
\geometry
{
    inner          = 1.1in,
    outer          = 1.1in,
    top            = 1.2in,
    bottom         = 1.2in,
    marginparwidth = 1.3in,
    marginparsep   = 0pt,
    headheight     = 21pt
}

% 段落首行缩进
\usepackage{indentfirst}

% 屏蔽宏包警告
\RequirePackage[]{silence}
\WarningsOff[hyperref]

% 设置书签、超链接等
\usepackage{hyperref}
\hypersetup{
    bookmarks = true,
    colorlinks,
    linkcolor = {black},
    citecolor = {blue!50!black},
    urlcolor  = black,
    bookmarksdepth = 4,
    pdfpagelabels
}

% 设置表格
\usepackage{tabu}
\usepackage{longtable}

% 设置列表格式
\usepackage{enumitem}
\setenumerate[1]{itemsep = 0pt, partopsep = 0pt, parsep = 0pt, topsep = 0pt, label = (\arabic*), leftmargin = \parindent}
\setitemize[1]  {itemsep = 0pt, partopsep = 0pt, parsep = 0pt, topsep = 0pt, leftmargin = 1em}
\setitemize[2]  {itemsep = 0pt, partopsep = 0pt, parsep = 0pt, topsep = 0pt, leftmargin = 1em}
\setdescription {labelindent = 0em, itemsep = 0pt, parsep = 0pt, topsep = 0pt, partopsep = 0pt, labelsep* = 1em, leftmargin = !, style = standard, font = \normalfont}

% 设置页眉页脚
\usepackage{fancyhdr}
\usepackage{lastpage}
\makeatletter
\DeclareDocumentCommand\authorname{m}{
    \gdef\qiqi@authorname{#1}
}
\DeclareDocumentCommand\maintitle{m}{
    \gdef\qiqi@title{#1}
}
\DeclareDocumentCommand\abstract{+m}{
    \long\gdef\qiqi@abstract{#1}
}

\fancypagestyle{plain}{%
    \fancyhf{}
    \fancyhead[L]{}
    \fancyhead[C]{\qiqi@title}
    \fancyhead[R]{\qiqi@authorname{}}
    \fancyfoot[C]{}
    \fancyfoot[R]{第 \thepage{} 页,共 \pageref{LastPage} 页}
    \fancyfoot[L]{\zhtoday}
    \renewcommand{\headrulewidth}{0.4pt}
    \renewcommand{\footrulewidth}{0.4pt}
}

\fancypagestyle{cover}{%
    \fancyhf{}
    \fancyhead[L]{}
    \fancyhead[C]{\qiqi@title}
    \fancyhead[R]{\qiqi@authorname{}}
    \fancyfoot[C]{}
    \fancyfoot[R]{第 \thepage{} 页}
    \fancyfoot[L]{\zhtoday}
    \renewcommand{\headrulewidth}{0.4pt}
    \renewcommand{\footrulewidth}{0.4pt}
}
\makeatother

\usepackage{pifont}
\let\proofname\relax
\usepackage[amsmath, amsthm, thmmarks, hyperref]{ntheorem}
\newcounter{DefinitionCounter}
\numberwithin{DefinitionCounter}{section}
\theoremstyle{plain}
\theoremsymbol{\ding{117}}
\theoremseparator{.}
\newtheorem{Definition}[DefinitionCounter]{定义}
\let\oldDefinition\Definition
\def\Definition{\oldDefinition\normalfont\kaishu}

\newcounter{CriterionCounter}
\numberwithin{CriterionCounter}{section}
\theoremstyle{plain}
\theoremsymbol{\ding{117}}
\theoremseparator{.}
\newtheorem{Criterion}[CriterionCounter]{判据}
\let\oldCriterion\Criterion
\def\Criterion{\oldCriterion\normalfont\itshape}

\newcounter{LemmaCounter}
\numberwithin{LemmaCounter}{section}
\theoremstyle{plain}
\theoremsymbol{\Circle}
\theoremseparator{.}
\newtheorem{Lemma}[LemmaCounter]{引理}
\let\oldLemma\Lemma
\def\Lemma{\oldLemma\normalfont\itshape}

\newcounter{ExampleCounter}
\numberwithin{ExampleCounter}{section}
\theoremstyle{plain}
\theoremsymbol{$\Box$}
\theoremseparator{.}
\newtheorem{Example}[ExampleCounter]{例}
\let\oldExample\Example
\def\Example{\oldExample\normalfont\itshape}

\newcounter{TheoremCounter}
\numberwithin{TheoremCounter}{section}
\theoremstyle{plain}
\theoremsymbol{$\Diamond$}
\theoremseparator{.}
\newtheorem{Theorem}[TheoremCounter]{定理}
\let\oldTheorem\Theorem
\def\Theorem{\oldTheorem\normalfont\itshape}

\newcounter{ProofCounter}
\numberwithin{ProofCounter}{section}
\theoremstyle{plain}
\theoremsymbol{\ding{110}}
\theoremseparator{.}
\newtheorem{Proof}[ProofCounter]{证明}
\let\oldProof\Proof
\def\Proof{\oldProof\normalfont\itshape}


% 设置图表标题格式
\usepackage{caption}
\captionsetup[figure]{
    font = {small, bf},
    labelsep = period,
    textformat = simple,
    skip = 5pt
}
\captionsetup[table]{
    font = {small, bf},
    labelsep = period,
    textformat = simple,
    skip = 5pt
}

% 设置封面、摘要和目录
\makeatletter
\AtBeginDocument{
    % page 1
    \pagestyle{empty}
    \pagenumbering{Alph}
    \pdfbookmark[1]{封面}{Cover}
    \begin{centering}
    \rule{0pt}{0pt}\\[100pt]
    {\Huge\bfseries \qiqi@title}\\[70pt]
    {\large\textbf{作者:}\qiqi@authorname}\\\vfill
    {\today}\\
    \end{centering}
    
    % page 2
    \clearpage
    \setcounter{page}{1}
    \renewcommand{\thepage}{\Roman{page}}
    % \pdfbookmark[1]{摘要}{Abstract}
    \addcontentsline{toc}{section}{摘要}
    \section*{摘要}
    \qiqi@abstract
    
    % page 3
    \clearpage
    \setcounter{tocdepth}{4}
    \pdfbookmark[1]{\contentsname}{Contents}
    \tableofcontents
    
    % page 4
    \clearpage
    \pagestyle{plain}
    \setcounter{page}{1}
    \renewcommand{\thepage}{\arabic{page}}
}
\makeatother

% 设置引用格式
\usepackage{cleveref}
\makeatletter
\crefformat{figure}              {图 #2#1#3}
\crefformat{section}             {第 #2#1#3 小节}
\crefformat{table}               {表 #2#1#3}
\crefformat{appendix}            {附录 #2#1#3}
\crefformat{equation}            {公式 #2(#1)#3}
\crefformat{Example}             {例 #2#1#3}
\crefformat{tcb@cnt@codebox}     {代码 #2#1#3}
\crefformat{tcb@cnt@codefilebox} {代码 #2#1#3}
\makeatother

% 设置行间距
\usepackage{setspace}

% 随机产生文本
\usepackage{lipsum}

% 设置行内代码
\usepackage[many, minted]{tcolorbox}
\tcbuselibrary{breakable}
\usepackage{minted}
\usepackage{color}
\makeatletter
\color{black}
\let\default@color\current@color
\def\inlinebox@true{true}
\define@key{inlinebox}{frame rule color} {\def\inlinebox@framerulecolor{#1}}
\define@key{inlinebox}{frame back color} {\def\inlinebox@framebackcolor{#1}}
\define@key{inlinebox}{frame text color} {\def\inlinebox@frametextcolor{#1}}
\define@key{inlinebox}{frame rule width} {\def\inlinebox@framerulewidth{#1}}
\define@key{inlinebox}{banner width}     {\def\inlinebox@bannerwidth{#1}}
\define@key{inlinebox}{show banner}[true]{\def\inlinebox@showbanner{#1}}
\define@key{inlinebox}{banner text color}{\def\inlinebox@bannertextcolor{#1}}
\define@key{inlinebox}{banner back color}{\def\inlinebox@bannerbackcolor{#1}}
\define@key{inlinebox}{banner text}      {\def\inlinebox@bannertext{#1}}
\NewDocumentCommand{\inlinebox}{O{} m}{%
  \setkeys{inlinebox}{%
    frame rule color  = black,
    frame back color  = white,
    frame text color  = black,
    frame rule width  = 0.4pt,
    banner width      = 8pt,
    show banner       = false,
    banner text color = white,
    banner back color = black,
    banner text       = BAN,
    #1
  }%
  \tcbox[%
    enhanced,
    tcbox raise base,
    nobeforeafter,
    boxrule           = \inlinebox@framerulewidth,
    top               = -1pt,
    bottom            = -1pt,
    right             = -1pt,
    arc               = 1pt,
    left              = \ifx\inlinebox@showbanner\inlinebox@true\inlinebox@bannerwidth-2pt\else-1pt\fi,
    colframe          = \inlinebox@framerulecolor,
    coltext           = \inlinebox@frametextcolor,
    colback           = \inlinebox@framebackcolor,
    before upper      = {\vphantom{蛤dg}},
    overlay           = {%
      \begin{tcbclipinterior} \ifx\inlinebox@showbanner\inlinebox@true
          \fill[\inlinebox@bannerbackcolor] (frame.south west) rectangle node[text = \inlinebox@bannertextcolor, scale = 0.4, font = \sffamily\bfseries, rotate = 90] {\inlinebox@bannertext} ([xshift = \inlinebox@bannerwidth]frame.north west);
        \fi
      \end{tcbclipinterior}%
    }%
  ]{#2}%
}
\makeatother
\newcommand{\inlinetext}[1]{\inlinebox{\mintinline{text}{#1}}}
\newcommand{\inlinepython}[1]{\inlinebox{\mintinline{python}{#1}}}
\newcommand{\inlinelatex}[1]{\inlinebox{\mintinline{latex}{#1}}}

% 设置行间代码
\usemintedstyle[python]{default}
\newtcblisting[auto counter, number within = section]{codebox}[1][]{%
  listing only,
  enhanced,
  breakable,
  sharp corners,
  left            = 6mm,
  top             = 0mm,
  bottom          = 0mm,
  colframe        = white,
  borderline horizontal = {0.4pt}{0pt}{black},
  colback         = white,
  colbacktitle    = white,
  coltitle        = black,
  titlerule       = 0.4pt,
  titlerule style = black,
  minted options  = {linenos, numbersep = 8pt, fontsize = \footnotesize, baselinestretch = 1.2, mathescape, breaklines = true, numbersep = 6.5pt},
  overlay         = {%
      \begin{tcbclipinterior}\draw[black] ([xshift = 6mm]frame.south west) -- ([xshift = 6mm]frame.north west);\end{tcbclipinterior}
    },
  #1%
}

\makeatletter
\def\codebox@caption{}
\def\codebox@label{}
\def\codebox@continuousnumber{false}
\def\codebox@true{true}
\def\codebox@language{python}
\define@key{codebox}{caption}{\renewcommand*{\codebox@caption}{#1}}
\define@key{codebox}{label}{\renewcommand*{\codebox@label}{#1}}
\define@key{codebox}{language}{\renewcommand*{\codebox@language}{#1}}
\define@key{codebox}{continuous number}[true]{\renewcommand*{\codebox@continuousnumber}{#1}}

\let\oldcodebox\codebox
\RenewDocumentCommand{\codebox}{O{}}{%
  \setkeys{codebox}{#1}%
  \oldcodebox[title = \hspace{-6mm}\small\bfseries 代码 \thetcbcounter. \codebox@caption, label = \codebox@label, minted language = \codebox@language]%
}

\let\oldendcodebox\endcodebox
\renewcommand{\endcodebox}{\oldendcodebox\noindent}

\let\latexbox\codebox
\let\endlatexbox\endcodebox
\makeatother

% 设置标题和作者
\maintitle{\LaTeX 模板}
\authorname{张琦}
\abstract{
基于 \inlinetext{ctex} 的一个模板,字体、封面、目录、书签、页眉、页脚等都已经设置好。方便直接使用。
}

\begin{document}
\section{模板说明}
\subsection{章节结构}
本模板定义了三个目录层级, 分别是一级目录 \inlinelatex{\section}, 二级目录 \inlinelatex{\subsection} 和三级目录 \inlinelatex{\subsubsection}. 目录的编号样式为阿拉伯数字加半角句号. 如果想要定义更深的目录层级, 模板作者给出两条建议:%
%
\begin{itemize}
  \item 仔细斟酌文章结构;
  \item 自行添加 \inlinelatex{\subsubsubsection} 和 \inlinelatex{\subsubsubsubsection} 的定义.
\end{itemize}

\subsubsection{字体设置}
本模板提供常用的文本样式:%
%
\begin{itemize}
  \item 请用 \inlinelatex{\textbf{粗体字, bold text}} 来插入\textbf{粗体字, bold text};
  \item 请用 \inlinelatex{\textit{斜体字, italic text}} 来插入\textit{斜体字, italic text};
  \item 请用 \inlinelatex{\texttt{等宽字, mono text}} 来插入\texttt{等宽字, mono text};
  \item 请用 \inlinelatex{\inlinetext{抄写, verbatim}} 来插入\inlinetext{抄写, verbatim};
  \item 请用 \inlinelatex{\inlinepython{func = lambda x: x**2}} 来插入 python 代码 \inlinepython{func = lambda x: x**2};
\end{itemize}

\subsection{插入公式}
公式分为行间公式和行内公式, 所谓行内公式就是在段落内的公式, 如 $a$, $b$, $x + y = z$ 等等. 行内公式由一对 \inlinetext{$} 符号包裹, 如\cref{code:inline equation} 所示.%
%
\begin{latexbox}[
  language = latex,
  caption = 行内公式代码,
  label = code:inline equation
]
公式分为行间公式和行内公式, 所谓行内公式就是在段落内的公式, 如 $a$, $b$, $x + y = z$ 等等. 行内公式由一对 \inlinetext{$} 符号包裹, 如\cref{code:inline equation} 所示.
\end{latexbox}

行间公式又分为无编号行间公式以及有编号行间公式. 无编号行间共识如下所示.%
%
\begin{equation*}
  \sum_{i=1}^\infty \frac{1}{i^2} = \frac{1}{1^2} + \frac{1}{2^2} + \frac{1}{3^2} + \cdots = \frac{\pi}{6}\text{.}
\end{equation*}%
%
其代码如\cref{code:between line equation without number} 所示.%
%
\begin{latexbox}[
  language = latex,
  caption = 无编号行间公式,
  label = code:between line equation without number
]
\begin{equation*}
  \sum_{i=1}^\infty \frac{1}{i^2} = \frac{1}{1^2} + \frac{1}{2^2} + \frac{1}{3^2} + \cdots = \frac{\pi}{6}\text{.}
\end{equation*}
\end{latexbox}
%
而有编号行间公式如\cref{eqn:between line equation with number} 所示.%
%
\begin{equation}\label{eqn:between line equation with number}
  \sin^2\theta + \cos^2\theta = 1\text{.}
\end{equation}%
%
其代码如\cref{code:between line equation with number} 所示.%
%
\begin{latexbox}[
  language = latex,
  caption = 有编号行间公式,
  label = code:between line equation with number
]
\begin{equation}\label{eqn:between line equation with number}
  \sin^2\theta + \cos^2\theta = 1\text{.}
\end{equation}
\end{latexbox}
%
有编号行间公式的引用可以使用 \inlinelatex{\cref{code:between line equation with number}} 进行引用. 其它的公式输入可以参考文献 \cite{lshort}.

\subsection{插入代码}
插入代码请使用 \inlinetext{codebox} 环境, \inlinetext{codebox} 环境有 3 个可配置参数, 分别是:%
%
\begin{itemize}
  \item \inlinetext{language}, 指定代码语言, \inlinetext{codebox} 环境会根据不同的语言实现不同的高亮, 默认值为 \inlinetext{python}.
  \item \inlinetext{caption}, 指定代码的标题, 默认值为空.
  \item \inlinetext{label}, 指定代码的引用键值, 默认值为空, 如果 \inlinetext{label} 为空, 则该代码不可被引用.
\end{itemize}

\begin{Example}\label{exp:insert code}\normalfont
python 中的单例模式如\cref{code:singleton in python} 所示.

\begin{codebox}[
    caption = python 单例模式,
    label = code:singleton in python
]
class Singleton(object):
    def __new__(cls, *args, **kw):
        if not hasattr(cls, '_instance'):
            orig = super(Singleton, cls)
            cls._instance = orig.__new__(cls, *args, **kw)
        return cls._instance
\end{codebox}
\end{Example}

要实现\cref{exp:insert code} 中的效果, 可以采用\cref{code:example of inserting code} 中的代码.%
%
\begin{latexbox}[
  language = latex,
  caption = 插入代码举例,
  label = code:example of inserting code]
python 中的单例模式如\cref{code:singleton in python} 所示.

\begin{codebox}[
    caption = python 单例模式,
    label = code:singleton in python
]
class Singleton(object):
    def __new__(cls, *args, **kw):
        if not hasattr(cls, '_instance'):
            orig = super(Singleton, cls)
            cls._instance = orig.__new__(cls, *args, **kw)
        return cls._instance
\end{codebox}
\end{latexbox}

\subsection{插入图片}

\subsection{插入表格}

\section{使用方法}


行内公式 $\int_{x = 0}^1 x^2 \mathrm{d}x$, 行间公式如下所示.%
%
\begin{equation*}
    x^2 + y^2 = z^2\text{.}
\end{equation*}%
%
有编号的行间公式如\cref{eqn:test} 所示.%
%
\begin{equation}
  \label{eqn:test}
  \sum_{i=1}^\infty \frac{1}{i^2} = \frac{\pi}{6}\text{.}
\end{equation}

\lipsum[1-2]

\begin{Definition}
  我就是我, 我不是别人, 别人也不是我.
\end{Definition}

\lipsum[5]

插图如\cref{fig:example of figure} 所示.%
%
\begin{figure}[!htb]
  \centering
  \begin{tikzpicture}
    \node[rectangle, draw, inner sep = 1pt, minimum size = 3cm, align = flush center] {23333}; 
  \end{tikzpicture}
  \caption{插图示例}
  \label{fig:example of figure}
\end{figure}

表格如\cref{tab:example of table} 所示.%
%
\begin{table}[!htb]
  \centering
  \tabulinesep=3pt
  \caption{表格示例}
  \label{tab:example of table}
  \begin{tabu}to 4cm{X[-1, c]|X[1, l]}
    \tabucline[1pt]{-}
       序号 & 内容  \\
    \tabucline[1pt]{-}
       1 & 锄禾日当午 \\
       2 & 汗滴禾下土 \\
       3 & 谁知盘中餐 \\
       4 & 粒粒皆辛苦 \\
    \tabucline[1pt]{-}
  \end{tabu}
\end{table}

\clearpage
\phantomsection
\addcontentsline{toc}{section}{参考文献}
\bibliographystyle{ieeetr}
\bibliography{./references}
\end{document}
