\documentclass{./template/article}

% 设置标题和作者
\maintitle{\LaTeX 模板}
\authorname{张琦}
\abstract{
    基于 \inlinetext{ctex} 的一个模板, 字体/封面/目录/书签/页眉/页脚等都已经设置好, 方便直接使用.
}

\begin{document}
\section{模板说明}
\subsection{章节结构}
本模板定义了三个目录层级, 分别是一级目录 \inlinelatex{\section}, 二级目录 \inlinelatex{\subsection} 和三级目录 \inlinelatex{\subsubsection}. 目录的编号样式为阿拉伯数字加半角句号. 如果想要定义更深的目录层级, 模板作者给出两条建议:%
%
\begin{itemize}
  \item 仔细斟酌文章结构;
  \item 自行添加 \inlinelatex{\subsubsubsection} 和 \inlinelatex{\subsubsubsubsection} 的定义.
\end{itemize}

\subsection{字体设置}
本模板提供常用的文本样式:%
%
\begin{itemize}
  \item 请用 \inlinelatex{\textbf{粗体字, bold text}} 来插入\textbf{粗体字, bold text};
  \item 请用 \inlinelatex{\textit{斜体字, italic text}} 来插入\textit{斜体字, italic text};
  \item 请用 \inlinelatex{\texttt{等宽字, mono text}} 来插入\texttt{等宽字, mono text};
  \item 请用 \inlinelatex{\inlinetext{抄写, verbatim}} 来插入\inlinetext{抄写, verbatim};
  \item 请用 \inlinelatex{\inlinepython{func = lambda x: x**2}} 来插入 python 代码 \inlinepython{func = lambda x: x**2};
\end{itemize}

\subsection{插入公式}
公式分为行间公式和行内公式, 所谓行内公式就是在段落内的公式, 如 $a$, $b$, $x + y = z$ 等等. 行内公式由一对 \inlinetext{$} 符号包裹, 如\cref{code:inline equation} 所示.%
%
\begin{latexbox}[
  language = latex,
  caption = 行内公式代码,
  label = code:inline equation
]
公式分为行间公式和行内公式, 所谓行内公式就是在段落内的公式, 如 $a$, $b$, $x + y = z$ 等等. 行内公式由一对 \inlinetext{$} 符号包裹, 如\cref{code:inline equation} 所示.
\end{latexbox}

行间公式又分为无编号行间公式以及有编号行间公式. 无编号行间共识如下所示.%
%
\begin{equation*}
  \sum_{i=1}^\infty \frac{1}{i^2} = \frac{1}{1^2} + \frac{1}{2^2} + \frac{1}{3^2} + \cdots = \frac{\pi}{6}\text{.}
\end{equation*}%
%
其代码如\cref{code:between line equation without number} 所示.%
%
\begin{latexbox}[
  language = latex,
  caption = 无编号行间公式,
  label = code:between line equation without number
]
\begin{equation*}
  \sum_{i=1}^\infty \frac{1}{i^2} = \frac{1}{1^2} + \frac{1}{2^2} + \frac{1}{3^2} + \cdots = \frac{\pi}{6}\text{.}
\end{equation*}
\end{latexbox}
%
而有编号行间公式如\cref{eqn:between line equation with number} 所示.%
%
\begin{equation}\label{eqn:between line equation with number}
  \sin^2\theta + \cos^2\theta = 1\text{.}
\end{equation}%
%
其代码如\cref{code:between line equation with number} 所示.%
%
\begin{latexbox}[
  language = latex,
  caption = 有编号行间公式,
  label = code:between line equation with number
]
\begin{equation}\label{eqn:between line equation with number}
  \sin^2\theta + \cos^2\theta = 1\text{.}
\end{equation}
\end{latexbox}
%
有编号行间公式的引用可以使用 \inlinelatex{\cref{code:between line equation with number}} 进行引用. 其它的公式输入可以参考文献 \cite{lshort}.

\subsection{插入代码}
插入代码请使用 \inlinetext{codebox} 环境, \inlinetext{codebox} 环境有 3 个可配置参数, 分别是:%
%
\begin{itemize}
  \item \inlinetext{language}, 指定代码语言, \inlinetext{codebox} 环境会根据不同的语言实现不同的高亮, 默认值为 \inlinetext{python}.
  \item \inlinetext{caption}, 指定代码的标题, 默认值为空.
  \item \inlinetext{label}, 指定代码的引用键值, 默认值为空, 如果 \inlinetext{label} 为空, 则该代码不可被引用.
\end{itemize}

\begin{Example}\label{exp:insert code}\normalfont
python 中的单例模式如\cref{code:singleton in python} 所示.

\begin{codebox}[
    caption = python 单例模式,
    label = code:singleton in python
]
class Singleton(object):
    def __new__(cls, *args, **kw):
        if not hasattr(cls, '_instance'):
            orig = super(Singleton, cls)
            cls._instance = orig.__new__(cls, *args, **kw)
        return cls._instance
\end{codebox}
\end{Example}

要实现\cref{exp:insert code} 中的效果, 可以采用\cref{code:example of inserting code} 中的代码.%
%
\begin{latexbox}[
  language = latex,
  caption = 插入代码举例,
  label = code:example of inserting code]
python 中的单例模式如\cref{code:singleton in python} 所示.

\begin{codebox}[
    caption = python 单例模式,
    label = code:singleton in python
]
class Singleton(object):
    def __new__(cls, *args, **kw):
        if not hasattr(cls, '_instance'):
            orig = super(Singleton, cls)
            cls._instance = orig.__new__(cls, *args, **kw)
        return cls._instance
\end{codebox}
\end{latexbox}

\subsection{插入图片}
在文件夹 \inlinetext{./figures/} 里, 有一幅 \LaTeX{} 吉祥物的插图 \inlinetext{mascots.pdf}, 可以用\cref{code:code of inserting figure} 将该图插入到文档中, 如\cref{fig:mascots of latex} 所示.%
%
\begin{figure}[!htb]
  \centering
  \includegraphics[width = 0.5\textwidth]{./figures/mascots.pdf}
  \caption{\LaTeX{} 吉祥物}
  \label{fig:mascots of latex}
\end{figure}

\begin{latexbox}[
  language = latex,
  caption = 插图图片代码示例,
  label = code:code of inserting figure
]
在文件夹 \inlinetext{./figures/} 里, 有一幅 \LaTeX{} 吉祥物的插图 \inlinetext{mascots.pdf}, 可以用\cref{code:code of inserting figure} 将该图插入到文档中, 如\cref{fig:mascots of latex} 所示.%
%
\begin{figure}[!htb]
  \centering
  \includegraphics[width = 0.5\textwidth]{./figures/mascots.pdf}
  \caption{\LaTeX{} 吉祥物}
  \label{fig:mascots of latex}
\end{figure}
\end{latexbox}

当然也可以用 \inlinetext{tikz} 宏包进行绘图, 利用\cref{code:example of tikz code} 可以产生如\cref{fig:petri net} 的插图. 如果对 \inlinetext{tikz} 有兴趣, 可以阅读 \cite{pgfmanual}.

\begin{figure}[!htb]
  \centering
  \begin{tikzpicture}[yscale=-1.6,xscale=1.5,thick,
  every transition/.style={draw=red,fill=red!20,minimum size=3mm},
  every place/.style={draw=blue,fill=blue!20,minimum size=6mm}]
  \foreach \i in {1,...,6} {
  \node[place,label=left:$p_\i$] (p\i) at (0,\i) {};
  \node[place,label=right:$q_\i$] (q\i) at (8,\i) {};
  }
  \foreach \name/\var/\vala/\valb/\height/\x in
  {m1/m_1/f/t/2.25/3,m2/m_2/f/t/2.25/5,h/\mathit{hold}/1/2/4.5/4} {
  \node[place,label=above:{$\var = \vala$}] (\name\vala) at (\x,\height) {};
  \node[place,yshift=-8mm,label=below:{$\var = \valb$}] (\name\valb) at (\x,\height) {};
  }
  \node[token] at (p1) {}; \node[token] at (q1) {};
  \node[token] at (m1f) {}; \node[token] at (m2f) {};
  \node[token] at (h1) {};
  \node[transition] at (1.5,1.5) {} edge [pre] (p1) edge [post] (p2);
  \node[transition] at (1.5,2.5) {} edge [pre] (p2) edge[pre] (m1f)
  edge [post](p3) edge[post] (m1t);
  \node[transition] at (1.5,3.3) {} edge [pre] (p3) edge [post] (p4)
  edge [pre and post] (h1);
  \node[transition] at (1.5,3.7) {} edge [pre] (p3) edge [pre] (h2)
  edge [post] (p4) edge [post] (h1.west);
  \node[transition] at (1.5,4.3) {} edge [pre] (p4) edge [post] (p5)
  edge [pre and post] (m2f);
  \node[transition] at (1.5,4.7) {} edge [pre] (p4) edge [post] (p5)
  edge [pre and post] (h2);
  \node[transition] at (1.5,5.5) {} edge [pre] (p5) edge [pre] (m1t)
  edge [post] (p6) edge [post] (m1f);
  \node[transition] at (1.5,6.5) {} edge [pre] (p6) edge [post] (p1.south east);
  \node[transition] at (6.5,1.5) {} edge [pre] (q1) edge [post] (q2);
  \node[transition] at (6.5,2.5) {} edge [pre] (q2) edge [pre] (m2f)
  edge [post] (q3) edge [post] (m2t);
  \node[transition] at (6.5,3.3) {} edge [pre] (q3) edge [post] (q4)
  edge [pre and post] (h2);
  \node[transition] at (6.5,3.7) {} edge [pre] (q3) edge [pre] (h1)
  edge [post] (q4) edge [post] (h2.east);
  \node[transition] at (6.5,4.3) {} edge [pre] (q4) edge [post] (q5)
  edge [pre and post] (m1f);
  \node[transition] at (6.5,4.7) {} edge [pre] (q4) edge [post] (q5)
  edge [pre and post] (h1);
  \node[transition] at (6.5,5.5) {} edge [pre] (q5) edge [pre] (m2t)
  edge [post] (q6) edge [post] (m2f);
  \node[transition] at (6.5,6.5) {} edge [pre] (q6) edge [post] (q1.south west);
  \end{tikzpicture}
  \caption{Petri-Net}
  \label{fig:petri net}
\end{figure}

\begin{latexbox}[
  language = latex,
  caption = \inlinetext{tikz} 插图代码示例,
  label = code:example of tikz code
]
% \usepackage{tikz}
% \usetikzlibrary{petri}
\begin{figure}[!htb]
  \centering
  \begin{tikzpicture}[yscale=-1.6,xscale=1.5,thick,
  every transition/.style={draw=red,fill=red!20,minimum size=3mm},
  every place/.style={draw=blue,fill=blue!20,minimum size=6mm}]
    \foreach \i in {1,...,6} {
    \node[place,label=left:$p_\i$] (p\i) at (0,\i) {};
    \node[place,label=right:$q_\i$] (q\i) at (8,\i) {};
    }
    \foreach \name/\var/\vala/\valb/\height/\x in
    {m1/m_1/f/t/2.25/3,m2/m_2/f/t/2.25/5,h/\mathit{hold}/1/2/4.5/4} {
    \node[place,label=above:{$\var = \vala$}] (\name\vala) at (\x,\height) {};
    \node[place,yshift=-8mm,label=below:{$\var = \valb$}] (\name\valb) at (\x,\height) {};
    }
    \node[token] at (p1) {}; \node[token] at (q1) {};
    \node[token] at (m1f) {}; \node[token] at (m2f) {};
    \node[token] at (h1) {};
    \node[transition] at (1.5,1.5) {} edge [pre] (p1) edge [post] (p2);
    \node[transition] at (1.5,2.5) {} edge [pre] (p2) edge[pre] (m1f)
    edge [post](p3) edge[post] (m1t);
    \node[transition] at (1.5,3.3) {} edge [pre] (p3) edge [post] (p4)
    edge [pre and post] (h1);
    \node[transition] at (1.5,3.7) {} edge [pre] (p3) edge [pre] (h2)
    edge [post] (p4) edge [post] (h1.west);
    \node[transition] at (1.5,4.3) {} edge [pre] (p4) edge [post] (p5)
    edge [pre and post] (m2f);
    \node[transition] at (1.5,4.7) {} edge [pre] (p4) edge [post] (p5)
    edge [pre and post] (h2);
    \node[transition] at (1.5,5.5) {} edge [pre] (p5) edge [pre] (m1t)
    edge [post] (p6) edge [post] (m1f);
    \node[transition] at (1.5,6.5) {} edge [pre] (p6) edge [post] (p1.south east);
    \node[transition] at (6.5,1.5) {} edge [pre] (q1) edge [post] (q2);
    \node[transition] at (6.5,2.5) {} edge [pre] (q2) edge [pre] (m2f)
    edge [post] (q3) edge [post] (m2t);
    \node[transition] at (6.5,3.3) {} edge [pre] (q3) edge [post] (q4)
    edge [pre and post] (h2);
    \node[transition] at (6.5,3.7) {} edge [pre] (q3) edge [pre] (h1)
    edge [post] (q4) edge [post] (h2.east);
    \node[transition] at (6.5,4.3) {} edge [pre] (q4) edge [post] (q5)
    edge [pre and post] (m1f);
    \node[transition] at (6.5,4.7) {} edge [pre] (q4) edge [post] (q5)
    edge [pre and post] (h1);
    \node[transition] at (6.5,5.5) {} edge [pre] (q5) edge [pre] (m2t)
    edge [post] (q6) edge [post] (m2f);
    \node[transition] at (6.5,6.5) {} edge [pre] (q6) edge [post] (q1.south west);
  \end{tikzpicture}
  \caption{Petri-Net}
  \label{fig:petri net}
\end{figure}
\end{latexbox}

\subsection{插入表格}
表格推荐使用 \inlinetext{tabu} 宏包, 如果要生成如\cref{tab:example of table} 所示的表格, 可以使用\cref{code:example of table} 进行插入.


\begin{table}[!htb]
    \centering
    \caption{表格示例}
    \label{tab:example of table}
    \begin{tabu}{c|l}
        \tabucline[1pt]{-}
        序号 & 文本 \\
        \hline
        1    & 锄禾日当午\\
        2    & 汗滴禾下土\\
        3    & 谁知盘中餐\\
        4    & 粒粒皆辛苦\\
        \tabucline[1pt]{-}
    \end{tabu}
\end{table}

\begin{latexbox}[
    language = latex,
    caption = 插入表格代码,
    label = code:example of table
]
\begin{table}[!htb]
    \centering
    \caption{表格示例}
    \label{tab:example of table}
    \begin{tabu}{c|l}
        \tabucline[1pt]{-}
        序号 & 文本 \\
        \hline
        1    & 锄禾日当午\\
        2    & 汗滴禾下土\\
        3    & 谁知盘中餐\\
        4    & 粒粒皆辛苦\\
        \tabucline[1pt]{-}
    \end{tabu}
\end{table}
\end{latexbox}

\section{使用方法}
首先安装 \LaTeX{}, 然后在控制台中运行 \inlinetext{xelatex -enable-write18 -syntex=1 main} 即可得到 \inlinetext{main.pdf} 文件.

\clearpage
\phantomsection
\addcontentsline{toc}{section}{参考文献}
\bibliographystyle{ieeetr}
\bibliography{./references}
\end{document}
